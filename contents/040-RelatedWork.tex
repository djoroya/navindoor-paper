\section{Estado del arte}\label{S2}
Los simuladores son una gran herramienta para mejorar el diseño, prueba y validación de sistemas de localización. Es por ello que podemos encontrar distintos intentos en la comunidad científica. A continuación se resumirá algunos trabajos existentes en el ámbito, además de describir sus principales características.

% ----------------------------------------------------
SMILe  \cite{Jankowski2018} es una herramienta de simulación de código abierto, para desarrollar y evaluar métodos de localización en interiores basados en el tiempo de propagación como métricas de localización, tales como el Tiempo de Vuelo (ToF) o la Diferencia de Tiempo de Llegada (TDoA). Esta herramienta está basada en el simulador OMNET+++ y un paquete de funcionalidades escritas en Python para el análisis y el procesamiento de datos. SMILe permite al usuario configurar un espacio, donde se pueden cambiar varios factores que afectan significativamente el rendimiento de la localización. Estos factores incluyen el despliegue de diferentes nodos, capacidades de radio y la inexactitud de relojes.
% Comparación Navindoor  

Tanto en Navindoor como en SMILe existen herramientas necesarias para el diseño, prueba y evaluación de algoritmos de localización a partir de ToF. Sin embargo, mientras que en SMILe señales de otra índole todavía no están contempladas, en Navindoor se abarcar más diversidad de tecnologías. Aun así, el modelo de simulación de señales de ToF de SMILe es más complejo que el modelo de simulación de Navindoor. 
% ----------------------------------------------------

PyLayers\cite{Amiot2013} es un simulador de radio frecuencia de código abierto. Se ha diseñado para evaluar el rendimiento de algoritmos de localización. El canal de radio se sintetiza mediante el uso de un  método de trazado de rayos basado en gráficos. El movimiento de personas se modela con un enfoque de fuerzas virtuales. Los datos simulados se pueden procesar directamente con uno de los algoritmos de localización incorporados o se pueden exportar a varias extensiones para el procesamiento externo. 
% Comparación Navindoor  

Al igual que en el caso anterior, PyLayers contiene un simulador complejo para una tecnología concreta. Aunque ofrece la posibilidad de procesar la señales dentro del propio simulador, hace énfasis en la exportación de sus resultados para el procesamiento externo. En Navindoor, se opta por ofrecer un interfaz para la incorporación de nuevos modelos de simulación, manteniendo todo el proceso en un mismo entorno de desarrollo.
% ----------------------------------------------------

% pro
\emph{Sensor Fusion and Tracking Toolbox} \cite{Mathworks} es una librería desarrollada por Mathworks que incluye algoritmos y herramientas para diseñar, simular y analizar sistemas que fusionan datos de varios sensores con el objetivo de hacer seguimiento de la posición y orientación de objetos.
Esta librería incluye funcionalidades para la generación de trayectorias y escenarios, simular mediciones de sensores inerciales (acelerómetro, giroscopio, magnetómetro), receptores de GPS, radar, sonar e infrarrojo, diferentes algoritmos de fusión y estimación (filtros de Kalman y filtros de partículas), y herramientas para visualizar, analizar y comparar el rendimiento de los diferentes algoritmos.

% Comparación Navindoor  
En está librería, al igual que en Navindoor, se provee de herramientas en todo el proceso de diseño de sistemas de localización. Sin embargo, en \emph{Sensor Fusion and Tracking Toolbox} por ahora solo se ha desarrollado una  CLI, careciendo de GUI. Por otra parte, los modelos de simulación en la generación de trayectorias son generales, no simulan un activo en concreto. En Navindoor los modelos de simulación están enfocados al movimiento de personas.