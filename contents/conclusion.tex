\section{Conclusión}\label{S5}

Se ha mostrado como los aspectos básicos de navindoor. Este software ha sido escrito para ser escalable, donde los propios usuarios puedan contribuir con nuevos modelos de simulación y de procesamiento. Aunque en este momento los modelos implentados son simples, su estructura modular es capaz de integrar modelos complejos. Además gracias a las APis de MATLAB para otros lenguajes de programación, es posible la integración de otros simuladores en distintos lenguajes, con una simple adaptación. Simluadores como los presentados en la sección \ref{S2}.

Es importante notar que la clase \emph{building} es una abstracción de la planimetría de un edificio, por lo que la traslación a otros formatos como \emph{XML} o \emph{JSON} es posible, y se están explorando por parte del equipo de desarrollo. Este desarrollo permitirá comunicaciones con plataformas como \emph{Open Street Maps}.

Navindoor es un software desarrollado en MATLAB, que mejora el desarrollo de algoritmos de posicionamiento. Esta constantemente en desarrollo y continuará agregando funcionalidades.