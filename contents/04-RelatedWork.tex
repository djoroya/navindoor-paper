\section{Estado del arte}
En la comunidad del posicionamiento se esta buscando una solución para estandarizar los desarrollos. Es por ello que se han realizado trabajos relacionados. En este apartado mencionaremos algunos trabajos y sus principales características.
% ----------------------------------------------------
\subsection{SMILe}
SMILe\cite{Jankowski2018} es un paquete escrito C++ y Python, dedicado a la simulación de señales \emph{ToF(Time of Flight)} y a su procesamiento. Este paquete soluciona la parte de definicion de esenario. En SMILe se puede definir las posiciones de los puntos de acceso y colocar paredes par que luego un modelo sofisticado pueda simular las señales recibidas por un objeto en movimiento. Por otro lado, el procesamiento de las señales esta escrito en python. SMILe se creo debido a la falta de herramientas de simulación de \emph{ToF}, por lo que por ahora no contiene otros tipos de señales, sin embargo se encuentra entre sus proximos desarrollos.  
% ----------------------------------------------------
\subsection{PyLayers}
PyLayers\cite{Amiot2013} es un software de simulacion centrado en la simulacion de \emph{RSS(Received Signal Strength)}, contiene modelos de señales muy realistas 
% ----------------------------------------------------
\subsection{Sensor Fusion and Tracking Toolbox}
\footnote{\url{https://es.mathworks.com/help/fusion/}}