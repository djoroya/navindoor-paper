

\subsection{Módulo de evaluación de algoritmos}

% E- Módulo de evaluación
% - Objetivo del módulo			
El objetivo de este último módulo es comparar distintas estimaciones de la trayectoria. La diferencia de las ejecuciones puede estar en el algoritmo utilizado para estimar, sin embargo también puede estar en los modelos de simulación de la trayectoria, en los modelos simulación de señales, o simplemente la frecuencia de muestreo de alguna señal. Deberemos notar que las estimaciones en Navindoor son objetos trayectorias por lo que desde el punto de vista del código, no existe diferencia estructural entre la trayectoria real y la estimación. Esto hace que la comparación entre ellas sea más fácil.
% - Explicación básica de la evaluación y comparación de rendimiento de los algoritmos

En la GUI, Navindoor nos muestra un lista de las trayectorias generadas, pudiendo seleccionar una de ellas. Para cada una de estas trayectorias se muestran las estimaciones disponibles. Podemos seleccionar las estimaciones que queramos para compararlas en un mismo gráfico representando la función de distribución acumulada empírica (eCDF), tomando como referencia la trayectoria real. De esta forma tenemos una visión del comportamiento de las dos estimaciones en un mismo gráfico.