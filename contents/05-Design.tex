\section{Arquitectura de Software}

El desarrollo de Navindoor se ha dividido en modulos claramente diferenciados. Estos son:
\begin{enumerate}
    \item Generación de planimetría 
    \item Generación trayectorias
    \item Generacion de señales
    \item Procesamiento de señales
    \item Comparación de métodos
\end{enumerate}

Dentro de cada uno de estos modulos se encuentran definidos clases de MATLAB que representan su version en la realidad. Un ejemplo muy claro esta en los objetos de la planimetría. En este modulo existen objetos que representan paredes, puertas, escaleras, etc. La idea de esto es crear un interfaz intuitiva para que el usuario correspondiente aprenda a utilizar Navindoor de forma progresiva.

\subsection{Planimetría}
\subsection{trayectorias}
\subsection{señales}
\subsection{Procesamiento}
\subsection{Comparación}