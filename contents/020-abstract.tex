\begin{abstract}
    
    En este trabajo se presenta la plataforma de simulación, Navindoor. Esta es una plataforma desarrollada en MATLAB para el diseño, prueba y evaluación de sistemas de localización, que provee de herramientas para la definición del escenario, la generación de trayectorias, la simulación de señales sintéticas, el procesamiento de estas señales y la comparación de algoritmos. Navindoor se ha diseñado de forma modular, de manera que los modelos de simulación y algoritmos de procesamiento sean independientes de la plataforma. De esta forma nuevos modelos y algoritmos pueden ser fácilmente implementados.  
    \newline 
\end{abstract}

\begin{IEEEkeywords}
localización, procesamiento de señales, simulador
\end{IEEEkeywords} 

 
%     En este artículo se presentará la herramienta navindoor. Esta es un toolbox para MATLAB dirigido a la investigación de sistemas de posicionamiento. Navindoor contiene herramientas necesarias para el desarrollo de nuevos algoritmos, tal como el diseño del escenario, la simulación de la trayectoria o generación de señales de radiofrecuencia, inerciales, entre otras. Navindoor ha sido diseñado de forma altamenete modular en donde el ususario puede desarrolar modelos de simulación o algoritmos de posicionamiento.