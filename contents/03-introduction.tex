\section{Introdución}

Gracias al desarrollo de IoT que se esta viendo acelerado estos últimos años, los algoritmos de localización estan experimentando un nuevo interés en el mundo tecnológico. Esto se debe a que nos estamos volviendo una máquina de recolección de señales, lo que hace que los método de localización sean tan variados como ecosistema de sensores. 

Aunque los algoritmos sean diversos, existe una serie de pasos comunes que se deben realizar antes que desarrollar un algoritmo de posicionamiento. Como puede ser la toma de medidas experimentales de las señales con la que se estimará la trayectoria, asi como medidas de la propia trayectoria. Esta recolección de datos debe ser variada, ya que la colección en pocas casuística puede agregar sesgos no deseados a nuestros algoritmos. Es por ello que lo ideal es realizar pruebas en varios entornos con distintas trayectorias. Esto hace que el tiempo de desarrollo se alarge. 

Navindoor soluciona este problema mediante la simulación del proceso de posicionamiento. El escenario, la trayectorias y las señales son simulados mediante modelos correspondientes. Además de permitir al usuario crear nuevos modelos si asi lo quisiese. De esta forma Navindoor se convierte en un simulador muy versatil, ya que dentro de ella se puede desarrollar tres tipos de algoritmos:  \cite{Correa2017} 
\begin{enumerate}
    \item \textbf{Algoritmos de simulación de trayectorias}: Los modelos de movimiento de un objetivo deberá reproducir las velocidades y aceleraciones. Este modelo puede cambiar si el objetivo es un robot o una peatón, es más el modelo de movimiento puede cambiar según el esenario.
    \item \textbf{Algoritmos de simulación de señales}: En Navindoor se contemplan dos tipos de señales, estas son las que dependen de  balizas para ser generadas y las que son solo dependientes de la trayectoria seguida (Seccion 1). Navindoor contiene modelos señales  \emph{RSS, ToF, AoA, Barometer,Magnetomer,etc.} que pueden ser modificados o creado otros a partir de ellos.
    \item \textbf{Algoritmos de posicionamiento}: Algoritmos de posicionamiento ya concidos como los filtros de Kalman ya estan implementados en Navindoor por defecto. Y de la misma forma que en los dos puntos anteriores, la implementación de nuevos algoritmos es muy fácil.
\end{enumerate}

Gracias a que la herramienta contiene modelos por defecto en cada uno de sus frentes, podemos desarrollar de forma independiente a los demás modelos. 

En este articulo, describiremos el estado del arte en materia de simladores, ademas de la arquitectura de software debajo de navindoor, la funcionalidad del software en algunos ejemplos concretos y por último la dirección de los futuros desarrollos.  

