\section{Conclusiones}\label{S5}

Se han mostrado las principales características de Navindoor.  Esta plataforma ha sido desarrollada para ser escalable, donde los propios usuarios puedan contribuir con modelos de simulación de trayectorias, modelos de señales y con algoritmos de localización. Aunque en este momento los modelos implementados son básicos, su estructura modular es capaz de integrar modelos complejos.
%  

A continuación mencionaremos los futuros desarrollos en la plataforma Navindoor.

Con respecto al módulo de la planimetría, es importante notar que la clase \emph{building} es una abstracción de un edificio, por lo que la traslación a otros formatos como \emph{XML} o \emph{JSON} es inmediata. Este desarrollo permitirá comunicaciones con plataformas como \emph{Open Street Maps}, permitiendo el uso de señales \emph{GNSS}, además de la fusión de tecnologías de localización en interiores con tecnologías de localización globales.
%

En cuanto a la simulación de trayectorias, se han implementado modelos de simulación de personas. Sin embargo esto puede generalizarse a otros tipos de activos como pueden ser los drones, robots, etc. Gracias a la fácil implementación de modelos, Navindoor es una plataforma muy versátil.

Por parte de la simulación de señales, gracias a las APIs de MATLAB para otros lenguajes de programación, es posible la integración de otros simuladores en distintos lenguajes. Simuladores como los presentados en la sección \ref{S2}, pueden ser integrados.

% 
Por último, se agregarán nuevas métricas para la comparación de trayectorias, siguiendo la misma filosofía modular tomada en los modelos de simulación y los algoritmos. De esta forma, podremos crear métricas de localización personalizadas e independientes a la plataforma.

% 
% Navindoor es una plataforma que mejora el proceso de diseño de sistemas de localización y esta constantemente en desarrollo. Este continuará agregando funcionalidades, que estarán disponibles en \cite{navindoor}.